\subsection{Notes on the derivatives of the gravitational potential}
\label{ssec:grav_derivatives}

The calculation of all the
$\mathsf{D}_\mathbf{n}(x,y,z) \equiv \nabla^{\mathbf{n}}\phi(x,y,z)$ terms up
to the relevent order can be quite tedious and it is beneficial to
automatize the whole setup. Ideally, one would like to have an
expression for each of these terms that is only made of multiplications
and additions of each of the coordinates and the inverse distance. We
achieve this by writing $\phi$ as a composition of functions
$\phi(u(x,y,z))$ and apply the \textit{Fa\`a di Bruno}
formula \citep[i.e. the ``chain rule'' for higher order derivatives,
 see e.g.][]{Hardy2006} to construct our terms:
\begin{equation}
\label{eq:faa_di_bruno}
\frac{\partial^n}{\partial x_1 \cdots \partial x_n} \phi(u)
= \sum_{A} \phi^{(|A|)}(u) \prod_{B \in
A} \frac{\partial^{|B|}}{\prod_{c\in B}\partial x_c} u(x,y,z),
\end{equation}
where $A$ is the set of all partitions of $\lbrace1,\cdots, n\rbrace$,
$B$ is a block of a partition in the set $A$ and $|\cdot|$ denotes the
cardinality of a set. For generic functions $\phi$ and $u$ this
formula yields an untracktable number of terms; an 8th-order
derivative will have $4140$ (!)  terms in the sum\footnote{The number
  of terms in the sum is given by the Bell number of the same
  order.}. \\ For the un-softened gravitational potential, we choose to write
\begin{align}
   \phi(x,y,z) &= 1 / \sqrt{u(x,y,z)}, \\
   u(x,y,z) &= x^2 + y^2 + z^2.
\end{align}
This choice allows to have derivatives of any order of $\phi(u)$ that
can be easily expressed and only depend on powers of $u$:
\begin{equation}
\phi^{(n)}(u) = (-1)^n\cdot\frac{(2n-1)!!}{2^n}\cdot\frac{1}{u^{n+\frac{1}{2}}},
\end{equation}
where $!!$ denotes the semi-factorial. More importantly, this
choice of decomposition allows us to have non-zero derivatives of $u$
only up to second order in $x$, $y$ or $z$. The number of non-zero
terms in eq. \ref{eq:faa_di_bruno} is hence drastically reduced. For
instance, when computing $\mathsf{D}_{(4,1,3)}(\mathbf{r}) \equiv
\frac{\partial^8}{\partial x^4 \partial y \partial z^3}
\phi(u(x,y,z))$, $4100$ of the $4140$ terms will involve at least one
zero-valued derivative (e.g. $\partial^3/\partial x^3$ or
$\partial^2/\partial x\partial y$) of $u$. Furthermore, among the 40
remaining terms, many will involve the same combination of derivatives
of $u$ and can be grouped together, leaving us with a sum of six
products of $x$,$y$ and $z$. This is generally the case for most of
the $\mathsf{D}_\mathbf{n}$'s and figuring out which terms are identical in a
given set of partitions of $\lbrace1,\cdots, n\rbrace$ is an
interesting exercise in combinatorics left for the reader \citep[see
  also][]{Hardy2006}. We use a \texttt{python} script based on this
technique to generate the actual C routines used within \swift. Some
examples of these terms are given in Appendix
\ref{sec:pot_derivatives}.

